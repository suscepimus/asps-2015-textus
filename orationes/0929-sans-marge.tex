% !TEX TS-program = LuaLaTex+se

\documentclass[12pt, twoside]{article} %%
\usepackage{fontspec} %%
%\usepackage[francais]{babel} %% si latin, faut régler la ponctuation, inutile avec Capella pour le moment
%%%%%%% taille de l'ouvrage
\usepackage[%
			paperwidth = 170 mm,% largeur de page
			%paperheight = 161 mm,% hauteur de page 9 lignes
			paperheight = 25 mm,% hauteur de page 1 ligne
			%paperheight = 36 mm,% hauteur de page 2 lignes
			%includehead,% inclure dans ces dimensions l'en-tête de page
			%headheight = 5 mm,% hauteur de l'en-tête
			%headsep = 4 mm,% séparation entre en-tête et page
			left = 0 mm,% marge interne
			right = 0 mm,% marge externe
			top = 1 mm,% marge haute
			bottom = 0 mm,% marge bas
			%margin= 0 mm% marges, option toutes égales
			]{geometry}% pour les formats
			
%\usepackage[a4paper, left=2cm, right=2cm, top=2cm, bottom=3cm]{geometry} %%margin=2 cm
\usepackage[autocompile]{gregoriotex} %%
\setmainfont[SmallCapsFont=Capella-SmallCaps, SmallCapsFeatures={Letters=SmallCaps}]{Capella} %
\pagestyle{empty} %%
\usepackage{graphicx}

%\raggedbottom

\newcommand\TitreOraison[1]{%
	\vspace{5mm plus 2mm minus 1mm}%
	\noindent{{\Large\color{gregoriocolor}#1}}%
	\vspace{2mm plus 2mm minus 0mm}%
	}%

\begin{document} %%

%\definecolor{gregoriocolor}{cmyk}{0,0,0,1}% pour avoir du noir uniquement

\def\GreStar{\textcolor{gregoriocolor}{*}}
\def\GreDagger{\textcolor{gregoriocolor}{†}}
\def\Rbar{\textcolor{gregoriocolor}{℟}}
\def\Vbar{\textcolor{gregoriocolor}{℣}}
\def\Abar{\textcolor{gregoriocolor}{Ⱥ}}
\gresetspecial{'oe}{ø}
\def\oea{ø} % oe accentué

%%%%%%% annotation et commentary
\gresetheadercapture{commentary}{grecommentary}{string}
%\grechangestyle{commentary}{\it\footnotesize\color{gregoriocolor}}
%\grechangedim{annotationraise}{-2.5 mm}{scalable}
%\grechangestyle{annotation}{\sc\small\color{gregoriocolor}}

\grechangestyle{firstword}{\sc} % à voir si à garder
\grechangestyle{initial}{\fontsize{54}{54}\selectfont}
%\greprintsigns{vepisema}{disable}
\gresetlinesbehindalteration{visible}


% space between glyphs in the same element
\grechangedim{interglyphspace}{0.0715 cm}{scalable}%
% space between elements which has the size of a note
\grechangedim{glyphspace}{0.143 cm plus 0 cm minus 0.01822 cm}{scalable}%
% minimal space between notes of different words
\grechangedim{interwordspacenotes}{0.3775 cm plus 0.005 cm minus 0.005 cm}{scalable}%
% minimal space between notes of the same syllable.
\grechangedim{intersyllablespacenotes}{0.2145 cm}{scalable}%
% minimal space between letters of different words. Makes sense to have
% the same plus and minus as interwordspacenotes.
\grechangedim{interwordspacetext}{0.17875 cm plus 0.286 cm minus 0.0143 cm}{scalable}%
% space between elements
%\grechangedim{interelementspace}{0.143 cm cm plus 0.00182 cm minus 0.00363 cm}{scalable}%
% larger space between elements
\grechangedim{largerspace}{0.286 cm plus 0.01822 cm minus 0.00911 cm}{scalable}%
% space before end-of-line custos
\grechangedim{spacebeforeeolcustos}{0.2145 cm}{scalable}%
% space after a clef at the beginning of a line
\grechangedim{spaceafterlineclef}{0.286 cm}{scalable}%
% bars having their own syllable, with text associated (new bar spacing algorithm only)
% plus or minus here will trigger some problems
\grechangedim{bar@virgula@standalone@text}{0.286 cm}{scalable}%
\grechangedim{bar@minima@standalone@text}{0.286 cm}{scalable}%
\grechangedim{bar@minor@standalone@text}{0.286 cm}{scalable}%
\grechangedim{bar@maior@standalone@text}{0.286 cm}{scalable}%
\grechangedim{bar@finalis@standalone@text}{0.286 cm}{scalable}%
% actual space before divisio finalis, not additional one
\grechangedim{bar@finalfinalis@standalone@text}{0.4 cm}{scalable}%
% bars having their own syllable, with no text associated (new bar spacing algorithm only)
\grechangedim{bar@virgula@standalone@notext}{0.286 cm}{scalable}%
\grechangedim{bar@minima@standalone@notext}{0.286 cm}{scalable}%
\grechangedim{bar@minor@standalone@notext}{0.286 cm}{scalable}%
\grechangedim{bar@maior@standalone@notext}{0.286 cm}{scalable}%
\grechangedim{bar@finalis@standalone@notext}{0.286 cm}{scalable}%
\grechangedim{bar@finalfinalis@standalone@notext}{0.286 cm}{scalable}%
% minimal space between letters of different syllable texts for text around bars
% (new bar spacing algorithm only)
\grechangedim{interwordspacetext@bars}{0.143 cm}{scalable}%
% minimal space between letters of different syllable texts for text around bars,
\grechangedim{interwordspacetext@bars@notext}{0.143 cm}{scalable}%
% rubber length that will be added around bars in new bar spacing algorithm
\grechangedim{bar@rubber}{0 cm plus 0.2145 cm minus 0.0143 cm}{scalable}%
% additional space that will appear around bars that are preceded by a custos and followed by a key.
\grechangedim{spacearoundclefbars}{0.03645 cm plus 0.00455 cm minus 0.0009 cm}{scalable}%
% Maximum offset between a bar and its associated text when the text goes left of the bar (new bar spacing algorithm only)
\grechangedim{maxbaroffsettextleft}{0.286 cm}{scalable}%
% Same as maxbaroffsettextleft when text goes right of the bar
\grechangedim{maxbaroffsettextright}{0.286 cm}{scalable}%
% Maximum offset between a no-bar (i.e. something like `text()` in gabc) and its associated text when the text goes left of the no-bar (new bar spacing algorithm only)
\grechangedim{maxbaroffsettextleft@nobar}{0.143 cm}{scalable}%
% Same as maxbaroffsettextleft@nobar when text goes right of the no-bar
\grechangedim{maxbaroffsettextright@nobar}{0.143 cm}{scalable}%
% Space between the two bars of a divisio finalis
% Maximum offset between a bar and its associated text when the text goes left of the bar and the bar terminates a line (i.e. something line `text(::z)` in gabc) (new bar spacing algorithm only)
\grechangedim{maxbaroffsettextleft@eol}{0.2145 cm}{scalable}%
% Same as maxbaroffsettextleft@eol when text goes right of the bar
\grechangedim{maxbaroffsettextright@eol}{0.2145 cm}{scalable}%
\grechangedim{divisiofinalissep}{0.1094 cm}{scalable}%
% maximal space between two syllables for which we consider a dash is not needed
\grechangedim{maximumspacewithoutdash}{0.2 mm}{scalable}%
% shift for a punctum mora for a note on a line
\grechangedim{linepunctummorashift}{-0.03 cm}{scalable}%
% shift for a punctum mora for a note in a space
\grechangedim{spacepunctummorashift}{-0.025 cm}{scalable}%
% shift for a punctum mora for the second note (in a space) of a pes with ambitus one
\grechangedim{spaceamonepespunctummorashift}{0.00183 cm}{scalable}%


%%% fonte grégorienne
\input{mechtildis.tex}

%the space above the lines
%\grechangedim{spaceabovelines}{0.35 cm}{scalable}%
% height that is added at the top of the lines if there is text above the lines (it must be bigger than the text for it to be taken into consideration)
%\grechangedim{abovelinestextraise}{-0.2 cm}{scalable}%

%\grechangedim{spacelinestext}{5 mm}{scalable}

% voir TEX pour l’Impatient n°161 page 139
\grechangedim{parskip}{2mm plus 1mm minus 0mm}{scalable}% espacement des partitions qui se suivent
%\grechangedim{lineskip}{2pt plus 2pt}{scalable}%
%\grechangedim{baselineskip}{21pt}{scalable}%espace de base entre les lignes de portée
%\grechangedim{lineskiplimit}{0pt}{scalable}% si ça va sous cette valeur TeX prend lineskip à la place

%\grechangedim{minimalinitialwidth}{1 cm}{scalable}
\fontsize{15}{15}\selectfont
\grechangestaffsize{21}


\gregorioscore{OS-Oremus}

\gregorioscore{OS-0929a}

\gregorioscore{OS-0929d}

\gregorioscore{OS-0929b}

\gregorioscore{OS-0929c}

\end{document}

